\documentstyle[11pt]{article}
\nofiles
\setlength{\textwidth}{6.5in}
\setlength{\oddsidemargin}{0in}
\setlength{\textheight}{9in}
\setlength{\topmargin}{-0.25in}

\setlength{\parindent}{0in}

\begin{document}
\section*{{\LARGE \sc Douglas William Nychka}}
\

\begin{tabular}{lr}
Institute for Mathematics Applied to Geosciences & \\
Computational and Information Systems Laboratory&  Voice:  (303) 497-1711 \\
National Center for Atmospheric Research  & Fax: (303) 497-2483 \\
P.O. Box 3000&  \verb+nychka@ucar.edu+ \\
Boulder, CO 80307-3000 & \verb+www2.image.ucar.edu/staff/nychka-doug+ 
\end{tabular}

\subsection*{Education}
\begin{tabular}{ll}
1978 B.A. &  Mathematics and Physics, Duke University  \\
1983  Ph.D. & Statistics,  University of
Wisconsin - Madison 
\end{tabular}
%%%%%%%%%%
\subsection*{Honors and Awards} 
\begin{tabular}{ll}
 1978&  {\it Summa cum laude}, Duke University\\
 1978& Julia Dale Mathematics Award  Duke University \\
 2003& Fellow, American Statistical Association \\
 2004& Jerry Sacks Award for Multidisciplinary Research  \\
 2013&   Distinguished Achievement Award ENVR Section 
 American Statistical Association \\
 2013 & Achievement Award, International Statistics and Climatology Meeting \\
 2015 & Fellow, Institute of Mathematical Statistics
\end{tabular}
%%%%%%%%%%%
\subsection*{Professional Appointments}
\begin{tabular}{ll}
8/97 - present & \begin{minipage} [t] {4.5in}
 National Center for Atmospheric Research, Boulder, CO. \\  
Visiting Scientist (8/97-7/99),
Senior Scientist  (8/99 - present),  Project leader (8/99 - 9/04) Geophysical Statistics Project, \\
 Director (10/04 - 2017), Institute for Mathematics Applied to 
Geosciences (IMAGe)
\end{minipage}\\
& \\
7/83 - 6/99 & \begin{minipage} [t] {4.5in} 
Department of Statistics, North Carolina State University, Raleigh, NC
Assistant (7/83 - 6/89),
Associate
(7/89 - 6/94) and Full Professor (7/94 - 7/99), 
\end{minipage}\\
& \\
1/94 - 7/03 &  \begin{minipage} [t] {4.5in}
National Institute of Statistical Sciences, Research
Triangle Park, NC,
Senior Fellow (1/94 -7/99)  and Trustee (2000 - 2003)
\end{minipage}\\
& \\
6/93 and 6/08 & \begin{minipage}[t]{4.5in}
Isaac Newton Institute for the Mathematical Sciences, Cambridge, England, Visiting Scholar
\end{minipage} \\
& \\
5/88 - 6/88, \  6/92 & \begin{minipage} [t] {4.5in}
 Statistics Group,
University of Bath, Bath, England, Visiting Faculty
\end{minipage}\\
& \\
7/89, \ 8/90 - 12/90 & \begin{minipage} [t] {4.5in}
 Operations Research and Industrial Engineering
Department,
 Cornell University, Visiting Faculty
\end{minipage}\\
& \\
%7/83 & \begin{minipage} [t] {4.0in}
%Postdoctoral Fellow, Department of Pathology\\
% University of Wisconsin, Madison
%\end{minipage}\\
%1981 - 6/83 & \begin{minipage} [t] {4.0in}
%Research Assistant, Department of Pathology\\
% University of Wisconsin, Madison
%\end{minipage}\\
%& \\
%1979 - 1980 & \begin{minipage} [t] {4.0in}
%Research Assistant, Department of Statistics
%University of Wisconsin, Madison
%\end{minipage}\\
%& \\
%1978, 1981 & \begin{minipage} [t] {4.0in}
%Teaching Assistant, Department of Statistics
% University of Wisconsin, Madison
%\end{minipage}\\
%& \\
%1978 & \begin{minipage} [t] {4.0in}
%Summer intern with Allied Corporation Statistics Group
%\end{minipage}\\
\end{tabular}
\newpage
%%%%%%%%%%%%%%
\subsection*{Professional service and memberships}
\begin{itemize}
\item Member, Program Committee, Year long program on Mathematical and Statistical Methods for Climate and the Earth System, Statistics and Applied Mathematical Sciences Institute, (2017 -- 2018)

\item Member, Scientific Advisory Board, Pacific Institute of Mathematics, University of British Columbia, CA (2015 -- )

\item Member, Scientific Advisory Board, European Union Surface Temperature for All Corners of Earth (EUSTACE), (2015 -- )

\item Member, Scientific Advisory Board, Canadian Statistical Sciences Institute, ( 2015 -- )

\item Member and Chair, Board of Governors, Institute for Mathematical Applications, 
University of Minnesota (2011-- 2015)

\item Member,  National Research Council Study on Verification \& Validation
 and Uncertainty Quantification, (2010-�2012)
 
\item Member, Committee on Surface Temperature Reconstructions for 
the Last 2,000 Years, National Research Council (2006)

\item Member and Chair, Scientific Advisory Panel,
 Center for Integrating Statistics and Environmental Sciences, 
University of Chicago (2003-- 2006)

\item Member, Committee on Applied and Theoretical Statistics (CATS), 
Board on Mathematical 
Sciences, The National Academies (2002 -- 2005)

\item Program Chair, 1994 Regional Institute of Mathematical Statistics Meetings, 
Birmingham, AL


\item Associate Editor  \\
{\it Technometrics} (1995-1998) \\
{\it Journal of Nonparametric Statistics} (1991--1995)\\
{\it Journal of the Royal Statistical Society Series B} (1992--1994)\\
{\it International Statistical Review} (1992--1994)\\
{\it Statistical Science} (1999--2003) \\
{\it Journal of the American Statistical Association} (2005 -- 2007 ) 

\item Reviewer \\
National Science Foundation, mail reviewer and panelist\\
{\it 
American Statistician,
Annals of Statistics,
Annals of Applied Statistics
Biometrics,
Econometrics,
Environmetrics, 
Journal of Climate,
Journal of the American Statistical Association,
Journal of Microscopy,
Letters in Statistics,
The Institute of Statistical Mathematics,
Nature,
Technometrics,
Spatial Statistics
}
\\

\item Memberships\\
{\it   American Statistical Association,
  Institute of Mathematical Statistics, 
  Society for Industrial and Applied Mathematics, 
  American Geophysical Union, American Meteorological Association,
  Amercian Meterological Association
 }
\end{itemize}

\subsection*{Peer-Reviewed  Publications}
\begin{enumerate}
\item K. Dalmasse , D. W. Nychka, S. E. Gibson 3, Y. Fan  and N. Flyer
(2016) ROAM: a Radial-basis-function Optimization Approximation Method
for diagnosing the three-dimensional  coronal magnetic field. {\it Frontiers in Stellar and Solar Physics}.

\item Alexeeff, S. E., Pfister, G. G., and Nychka, D. (2016). A Bayesian
  model for quantifying the change in mortality associated with future
  ozone exposures under climate change.
{\it Biometrics}, 72(1), 281-288.

\item Alexeeff, S. E., Nychka, D., Sain, S. R., and Tebaldi,
  C. (2016). Emulating mean patterns and variability of temperature
  across and within scenarios in anthropogenic climate change
  experiments. {\it Climatic Change}, 1-15.

\item Anderson, A. N., Browning, J. M., Comeaux, J., Hering, A. S., and
  Nychka, D. (2016).
 A comparison of automated statistical quality control methods for
 error detection in historical radiosonde temperatures. International
 {\it Journal of Climatology}, 36(1), 28-42.


 \item  Heaton, M. J., Katzfuss, M., Berrett, C.,  and  Nychka, D. W. (2015). Constructing valid spatial processes on the sphere using kernel convolutions. {\it Environmetrics.} 25 (1), 2-15. 

 \item Kleiber, W and DW Nychka (2015).  Equivalent kriging. {\it Spatial Statistics} 12 31-34.

\item Nychka,D.,  S. Bandyopadhyay, D. Hammerling, F. Lindgren, and S. Sain (2015).
 A Multi-resolution Gaussian process model for the analysis of large spatial data sets. 
  {\it Journal of Computational and Graphical Statistics} 24 (2) 579-599.
 
 
\item S. E. Tolwinski-Ward, M. P. Tingley, M. N. Evans, M. K. Hughes, D. W. Nychka. (2015). Probabilistic reconstructions of local temperature and soil moisture from tree-ring data with potentially time-varying climatic response. {\it Climate Dynamics}, 44, Page 791-806

\item Lombardozzi, D., G. Bonan and D. Nychka (2014) The emerging
 anthropogenic signal in land-atmosphere carbon cycle
 coupling, {\it Nature Climate Change}, 4, 796�800.



\item Anderes, E., R. Huser, D. Nychka and M. Coram (2013)
  Nonstationary positive definite tapering on the plane.  {\it Journal
    of Computational and Graphical Statistics} \\
    DOI: 10.1080/10618600.2012.729982.

\item  Benestad, R. E., D. Nychka, L. O. Mearns, (2012). Spatially and
  temporally consistent prediction  of heavy precipitation from mean
  values. {\it Nature Climate Change} 2, 544�547  doi:10.1038/nclimate1497

\item Benestad, R. E., D. Nychka, L. O. Mearns, (2012). Specification of wet-day daily rainfall quantiles from the mean value. {\it Tellus A }  64, 14981.

\item Kleiber, W. and D. Nychka (2012). Nonstationary Modeling for
  Multivariate Spatial Processes.{\it  Journal of Multivariate Analysis}. 
   112,  76�91.
  
\item  Sun, Y. M. Genton and D. Nychka (2012). 
Exact fast computation of band depth for large functional datasets: How quickly can one million curves be ranked?  {\it Stat} 1 ,  68�74. 

\item Matsuo, T., D. W. Nychka, and D. Paul (2011). Multi-resolution (wavelet) based non-stationary  covariance modeling 
for incomplete data: Smoothed Monte-Carlo EM approach. {\it Computational Statistics and Data Analysis}. 55, 2104-2113.

\item Oh, H-S, T. C. M. Lee, and D. W. Nychka (2011). Fast Nonparametric Quantile Regression with Arbitrary 
Smoothing Methods. {\it Journal of Computational and Graphical Statistics.},  20(2): 510-526.

\item    Li, B., D. W. Nychka and C. M. Ammann (2010). The Value of Multi-proxy Reconstruction of Past Climate.  
{\it Journal of the American Statistical Association.}  105(491): 883-895. doi:10.1198/jasa.2010.ap09379.

\item Sain S. , D. Nychka and L. Mearns (2010). Functional ANOVA and regional climate experiments: A statistical analysis of dynamic downscaling. {\it Environmetrics} 22, 700-711. 

\item Winter, C. L. and D. Nychka (2010) Forecasting skill of model averages. 
{\it Stochastic Environmental Research and Risk Assessment}, 24(5) 633-638.

\item Drignei, D., C.E. Forest, and D Nychka (2009).
 Parameter estimation for computationally intensive nonlinear
  regression with an application to climate modeling. 
   {\it  Annals of Applied Statistics} 2(4) 1217-1230.

\item Romero Lankao, P., J. L.Tribbia and D. Nychka (2009). Testing theories to explore the drivers of cities� atmospheric emissions. {\it Ambio}, 38, 236-244. 18.

\item Storlie, C.,  C.M. Lee, J. Hannig, and D. Nychka (2009). 
Tracking of multiple merging and splitting targets: A statistical perspective
 {\it Statistica Sinica} 1--31


\item Huang, J-C, K.V. Smith, D. Nychka (2008). Semi-parametric
  Discrete Choice Measures of Willingness to Pay. {\it Economics
    Letters}, DOI:10.1016/j.econlet.2008.06.010.

\item  Jun, M., R. Knutti, and D W. Nychka (2008).
 Local eigenvalue analysis of CMIP3 climate model errors. {\it Tellus A},  60, 992-1000.
%%%%%%%%%%%%stopped here
\item Jun, M., R. Knutti and D. Nychka. (2008). Spatial Analysis to 
Quantify Numerical Model Bias and Dependence: How Many Climate Models 
Are There? {\it Journal of the American Statistical Association}, 103, 
934-947.

\item  Kaufman, C., Schervish, M., and Nychka, D. (2008)
Covariance Tapering for Likelihood-Based Estimation in Large Spatial
Datasets {\it Journal of the American Statistical Association}
103(484): 1545-1555.

\item Khare S. P., Anderson J. L., Hoar T. J., Nychka D., (2008). An Investigation into the Application of an Ensemble Kalman Smoother to High-Dimensional Geophysical Systems. {\it Tellus A}. 60: 97-112.

\item 
 A. Malmberg, A. Arellano, D. P. Edwards, N. Flyer, D. Nychka, C. Wikle.
(2008) Interpolating Fields of Carbon Monoxide Data using a Hybrid
Statistical-Physical Model. {\it Annals of Applied
Statistics.}, 2(4), 1231-1248. 

\item Romero-Lankao, P., J. L. Tribbia and D. Nychka (2008) 
 Development and greenhouse gas emissions deviate from the �modernization� theory and �convergence� hypothesis. {\it Climate Research} 38, 17-29.

\item Santer, B., P. W. Thorne, L. Haimberger, K. E. Taylor, T. M. L. Wigley, J. R. Lanzante, S. Solomon, M. Free,  P. J. Gleckler, P. D. Jones, T. R. Karl, S. A. Klein, C. Mears, D. Nychka, G. A. Schmidt,  S. C. Sherwood,l and F. J. Wentz (2008) Consistency of modelled and observed temperature trends in the tropical troposphere. {\it International  Journal of Climatology}
 28: 1703�1722. 

\item Smith, R.L., C. Tebaldi, D. Nychka and L.O.Mearns
(2008). Bayesian Modeling of Uncertainty in Ensembles of Climate
Models.  {\it Journal of the American Statistical Association} 104, 97-116.

\item Whitcher, B., T. C. M. Lee, J. B. Weiss, 
T. J. Hoar and D. W. Nychka (2008). 
A Multiresolution Census Algorithm for Calculating Vortex Statistics in Turbulent Flows,
 {\it Journal of the Royal Statistical Society Series C}  (Applied Statistics), 57: 293�312.

\item Cooley, D.  D. Nychka, and P. Naveau. (2007) Bayesian Spatial
Modeling of Extreme Precipitation Return Levels. {\it Journal of
the American Statistical Association}, 102, 824-840.

 
\item Furrer, R., R. Knutti, S. R. Sain, D. W. Nychka, and G. A. Meehl
(2007), Spatial patterns of probabilistic temperature change
projections from a multivariate Bayesian analysis, {\it Geophysical
Research Letters}, 34, L06711, doi:10.1029/2006GL027754.

 \item Furrer, E. M.  and D. W. Nychka. (2007). A framework to
understand the asymptotic properties of Kriging and splines. {\it
Journal of the Korean Statistical Society} 36, 57-76.

\item Furrer, R, Sain, S. R., Nychka, D., and Meehl, G. A. (2007).
Multivariate Bayesian Analysis of Atmosphere-Ocean General Circulation
Models. {\it Environmental and Ecological Statistics}, 14(3), 249-266,
doi:10.1007/s10651-007-0018-z.

\item  Li, B., C. Ammann, D. Nychka. (2007).The Hockey Stick and the 1990s: A Statistical Perspective on
 Reconstructing Hemispheric Temperatures.  {\it Tellus}, 59, 591-598.

\item Oh, H-S, D. Nychka and T. Lee. (2007). The role of pseudodata
for robust smoothing with application to wavelet regression.  {\it
Biometrika} doi: 10.1093/biomet/asm064.

\item Fuentes, M., T.G.F. Kittel, and D. Nychka (2006) 
Sensitivity of ecological models to spatial-temporal estimation of
their climate drivers: Statistical ensembles for forcing. 
{\it Ecological Applications}, 16, pp. 99-116.

\item Furrer, R. M. Genton, and D. Nychka (2006) 
Covariance tapering for interpolation of large spatial datasets.
 {\it Journal of Computational and Graphical Statistics } 15(3), 502-523.

\item Sain, S.,  J. Shirkant, L. Mearns   and D. Nychka (2006). A multivariate spatial model for soil water profiles. 
 {\it Journal of Agricultural Biological and Environmental Statistics} 11(4) 462-480.

\item Bengtsson, T.,  R.F. Milliff, R Jones, D. Nychka, P.
Niiler (2005). A state space model for ocean drifter motions dominated by inertial oscillations. {\it Journal of Geophysical Research- Oceans}. 110, C10015, doi:10.1029/2004JC002850.

\item Feddema, J., K. Oleson, G. Bonan, L. Mearns, W. Washington, 
G. Meehl, and D. Nychka (2005). A Comparison of GCM response to 
historical anthropongenic land cover change and model sensitivity to 
uncertainty in present-day land cover representations. {\it Climate 
Dynamics}, 25, 581-609, DOI 10.1007/s00382-005-0038-z.

\item Gilleland, E.  and Nychka, D. (2005) Statistical Models for Monitoring and Regulating Ground-level Ozone {\it Environmetrics}. 16, 535--546. 

\item Tebaldi, C. R. L. Smith, D. Nychka, and L. Mearns (2005)
Quantifying Uncertainty in Projections of Regional Climate Change: A
Bayesian Approach to the Analysis of Multimodel Ensembles
{\it Journal of Climate},18, 1524--1540.

\item  Meehl, G.A., Tebaldi, C. and Nychka, D. (2004). Changes in 
frost days in simulations of 21st century climate. {\it Climate Dynamics}.

\item Oh, H-S, Nychka, D. , Brown, T. and Charbouneau, P. (2004). Periodic Analysis of Variable Stars by Robust Smoothing. {\it Journal of the Royal Statistical Society Series C}. {\bf 53}, 15--30.

\item Bengtsson, T., Snyder, C. and Nychka, D. (2003)
A nonlinear filter that extends to high dimensional systems 
{\it Journal of Geophysical Research -Atmosphere}, {\bf 108}, 1-10

\item Hoar, T., Milliff, R., Nychka, D., Wikle, C.,
Berliner, L.M. (2003). Winds from a Bayesian Hierarchical Model: Computation for Atmosphere-Ocean Research. {\it Journal of Computational and Graphical Statistics}, {\bf 12}, 781-807.

\item Holland, D., Cox, W.M., Scheffe, R., Cimorelli, A., Nychka, D. and Hopke, P.  (2003). Spatial Prediction of Air Quality Data. {\it Environmental Manager}, August.

\item Milliff, R.F., Niiler,P.P , Morzel, J., Sybrandy,A.E. Nychka, D.  and  Large, W.G.(2003). Mesoscale Correlation Length Scales from NSCAT and MiniMET Surface Wind Retrievals. {\it Journal of Atmospheric and Oceanic Technology}, {\bf 20}(4), 513-533.

\item Oh, H-S., Ammann, C.M.,  Naveau, P., Nychka,D. Otto-Bliesner, B. L. (2003) Multi-resolution time series analysis applied to solar irradiance and climate reconstructions. {\it Journal of Atmospheric and Solar-Terrestrial  Physics}, {\bf 2}, 191-201.

\item Matsuo, T., Nychka, D., Richmond, A. (2002). Modes of high-latitude electric field variability derived from DE-2 measurements: Empirical Orthogonal Function (EOF) Analysis. {\it Geophysical Research Letters}. {\bf 29} (7), doi:10.1029/2001GL014077.

\item {Nychka, D., Wikle, C.} and {Royle, J. A.} (2002).
 Multiresolution models for nonstationary spatial covariance functions. {\it Statistical Modelling}. {\bf 2}, 315-332. 

\item Tebaldi, C., Nychka, D., Brown, B.G. and Sharman, B.
(2002). Flexible discriminant techniques for forecasting clear-air
turbulence. {\it Environmetrics}, {\bf 13}, 859-878. 

\item  Cummins, D. J., Filloon, T. G. and Nychka, D. (2001).
Confidence Intervals for Nonparametric Curve Estimates: Toward more
Uniform Pointwise Coverage.{\it Journal of the American Statistical Association}, {\bf 96}, 233-246.

\item Johns, C. Nychka, D.,  Kittel, T., Daly, C. (2001). Infilling Sparse Records of Precipitation Fields. {\it Journal of the American Statistical Association}, {\bf 98}, 796--806.


\item {Small, E. E., Sloan, L. C.} and {Nychka, D.} (2001).
Changes in Surface Air Temperature Caused by Desiccation of the Aral
Sea. {\it Journal of  Climate}, {\bf 14}, 284-299.

\item Tebaldi, C., Branstator, G. and Nychka, D. (2001). Looking for
Nonlinearities in the Large Dynamics of the Atmosphere. {\it
Proceedings of  1$^{st}$ SIAM Conference on Scientific Data Mining.}

\item {Wikle, C., Milliff, R., Nychka, D.} and {Berliner, L. M.} (2001).
Spatiotemporal Hierarchical Bayesian Modeling: Tropical Ocean Surface
Winds. {\it Journal of  American Statistical Association}, {\bf 96}, 382-397.

\item  Davis, J. M., Nychka, D. and Bailey, B. (2000). A Comparison
of the Regional Oxidant Model with Observed Ozone Data. {\it Atmospheric Environment}, {\bf 34}, 2413-2423.

\item Errico, R. M., Fillion, L., Nychka, D. and Lu, Z.-Q. (2000).
Some Statistical Considerations Associated with the Data Assimilation of Precipitation Observations. {\it Quarterly Journal of the Royal Meteorological Society}, {\bf 126}, 339-359, Part A.

\item Huang, J.-C. and Nychka, D. (2000). A Nonparametric Multiple
Choice Model within the Random Utility Framework.  {\it Journal of
Econometrics}, {\bf 2}, 207-225.

\item  Santer, B. D., Wigley, T. M. L., Boyle, J. S., Gaffen, D. J.,
Hnilo, J. J., Nychka, D., Parker, D. E. and Taylor, K. E. (2000).
Statistical Significance of Trends and Trend Differences in
Layer-average Atmospheric Temperature Time Series. {\it Journal of Geophysical Research--Atmosphere}, {\bf 105}, 7337-7356.

\item  Tsai, K., Brownie, C., Nychka, D. W. and Pollock, K. H. (1999).
Smoothing Hazard Functions for Telemetry Survival Data in
Wildlife Studies. {\it Bird Study} {\bf 46}, 47-54, Suppl. S 1999.

\item {Davis, J. M., Eder, B., Nychka, D.} and {Yang, Q.} (1998).
Modeling the Effects of Meteorology on Ozone in Houston Using Cluster Analysis and Generalized Additive Models. {\it Atmospheric Environment}, {\bf 32}, 2505-2520.

\item {Ellner, S., Bailey, B., Bobashev, G., Gallant, A. R.,
Grenfell, B.} and {Nychka, D.} (1998). Noise and Nonlinearity in Measles Epidemics: Combining Mechanistic and Statistical Approaches to Population Modeling. {\it American Naturalist}, {\bf 151}, 425-440.


\item {Royle, J. A.} and {Nychka, D.} (1997).  An Algorithm for the
Construction of  Spatial Designs with an Implementation in SPLUS.
{\it Computers and Geosciences}, {\bf 24}, 479-488.



\item { Graham, M., Paulos, J.} and { Nychka, D.} (1995). Template-based MOSFET Device Model. { \it IEEE Transactions on CAD/ICAS}, {\bf 14}, 924-933.

\item {O'Connell, M.} and  { Nychka, D.} (1995).
A Generalized Linear Classification Model with a Smooth Link
Function and Predictors Obtained from Quantile Spline Fits to
High-Dimensional Data. {\it Journal of Statistical Planning and
Inference}, {\bf 45}, 153-164.

\item { Nychka, D.}  (1995).  Smoothing Splines as Locally
Weighted Averages. {\it Annals of Statistics}, {\bf 23}, 1175-1197.


\item { Nychka, D., Gray, G., Haaland, P., Martin, D.} and {O'Connell, M.} (1995). A Nonparametric Regression Approach to Syringe Grading
for Quality Improvement. {\it Journal of the American Statistical
Association}, {\bf 90}, 1171-1178.

\item  { Nychka, D.} and { Ruppert, D.}  (1995).  A Nonparametric Transformation for Both Sides of a Regression Model. {\it Journal of the Royal Statistical Society, Series B}, {\bf 57}, 519-532.

\item  { Haaland, P., McMillan, N., Nychka, D.} and {Welch, W.} (1994).
Analysis of Space-filling Designs. {\em Computing Science and
  Statistics: Computationally  Intensive Statistical
      Methods}, {\bf 26}, 111-120.

\item  { Meier, K.} and { Nychka, D.}  (1993).  Nonparametric
Estimation of Rate Equations for Nutrient Uptake.  {\it Journal
of The American Statistical Association}, {\bf 88}, 602-614.

\item  { Schluter, D.} and { Nychka, D.}  (1993).  Exploring
Fitness Surfaces. {\it American Naturalist}, {\bf 143}, 597-616.

\item  { Bloomfield, P.} and { Nychka, D.}  (1992).  Climate
Spectra and Detecting Climate Change.  {\it Climatic Change},
 {\bf 21}, 275-287.

\item  { Nychka, D., Ellner, S., McCaffrey, D.} and {Gallant,
A. R.}  (1992).  Finding Chaos in Noisy Systems.  {\it Journal
of the Royal Statistical Society Series B}, {\bf 54}, 399-426.

\item  { Ellner, S., Gallant, A. R., McCaffrey, D.} and {
Nychka, D.}  (1991).  Converge Rates and Data Requirements for
Jacobian-based Estimates of Lyapunov Exponents from Data.  {\it
Physics Letters}, {\bf 153}, 357-363.

\item  { Nychka, D.}  (1991).  Choosing a Range for the Amount
of Smoothing in Nonparametric Regression.  {\it Journal of the
American Statistical Association}, {\bf 86}, 653-664.

\item  { McCaffrey, D., Nychka, D., Ellner, S.} and { Gallant,
A. R.}  (1990). Estimating Lyapunov Exponents with Nonparametric Regression.  {\it Journal of the American Statistical
Association}, {\bf 87}, 682-695.

\item  { Nychka, D.}  (1990).  The Average Posterior Variance of
a Smoothing Spline and a Consistent Estimate of the Averaged
Squared Error.  {\it Annals of Statistics}, {\bf 18}, 415-428.

\item  { Nychka, D.}  (1990).  Some Properties of Adding a
Smoothing Step to the EM Algorithm. {\it Statistics and
Probability Letters}, {\bf 9}, 187-193.

\item  { Nychka, D., Ellner, S., McCaffrey, D.} and { Gallant,
A. R.}  (1990).  Statistics for Chaos.  {\it Statistical
Computing and Graphics Newsletter},  American Statistical
Association, Volume 1.


\item  { Silverman, B. W., Jones, M. C., Wilson, J. D.} and {
Nychka, D. W.}  (1990).  A Smoothed EM Approach to a Class of
Problems in Image Analysis and Integral Equations, with
discussion in {\it Journal of the Royal Statistical Society Series
B}, {\bf 52}, 271-324.

\item  { Nychka, D.} and { Cox, D. D.}  (1989).  Convergence
Rates for Regularized Solutions of Integral Equations from Discrete
Noisy Data.  {\it Annals of Statistics}, {\bf 17}, 556-572.

\item  { Nychka, D.}  (1988).  Bayesian `Confidence' Intervals
for Smoothing Splines.  {\it Journal of the American Statistical
Association}, {\bf 83}, 1134-1143.

\item { Gallant, A. R.} and { Nychka, D.}  (1987).
Semi-nonparametric Maximum Likelihood Estimation.  {\it
Econometrica}, {\bf 15}, 363-390.

\item { Nychka, D., Pugh, T., King, J., Koen, H., Wahba, G.,
Chover, J.} and { Goldfarb, S.}  (1984).  Optimal Use of
Sampled Tissue Sections for Estimating the Number of Hepatocellular
Foci.  {\it Cancer Research}, {\bf 44}, 178-183.

\item { Nychka, D., Wahba, G., Pugh, T. D.} and { Goldfarb,
S.}  (1984).  Cross-Validated Spline Methods for the Estimation
of Three-Dimensional Tumor Size Distributions from Observations on
Two-Dimensional Cross Sections.  {\it Journal of the American
Statistical Association}, {\bf 79}, 832-846.

\item { Koen, J., Pugh, T., Nychka, D.} and { Goldfarb, S.}
(1983).  Presence of $\beta$-fetoprotein Positive Cells in
Hepatocellular Foci Induced by Single Injection of
Dimethylnitrosamine in Infant Mice.  {\it Cancer Research}, {\bf
43}, 702-708.

\item { Pugh, T. D., King, J. H., Koen, H., Nychka, D., Chover,
J., Wahba, G., He, Y.} and { Goldfarb, S.}  (1983).  A
Reliable Mathematical Stereologic Method for Estimating the Number
of Hepatocellular Foci from their Transections.  {\it Cancer
Research}, {\bf 43}, 1261-1268.

\item { Reinsel, G., Tiao, G. C., Wang, M. N., Lewis, R.} and
{ Nychka, D.}  (1981).  Statistical Analysis of Stratospheric
Ozone Data for the Detection of Trends.  {\it Atmospheric
Environment}, {\bf 15}, 1569-1577.


\end{enumerate}

\subsection*{Manuscripts in review}
\begin{enumerate}
\item { Douglas Nychka, Dorit Hammerling, Mitchell Krock, Ashton Wiens, (2017)  }
Modeling and emulation of nonstationary Gaussian fields, 
{https://arxiv.org/abs/1711.08077}
\end{enumerate}

\subsection*{Book chapters, discussions and other edited  publications}
\begin{enumerate}

\item Nychka, D. and Wikle C. (2017). Spatial Analysis in Climatology. Chapter in  
{\it Handbook of Environmental  Statistics} ed. A Gelfand and   R Smith. Chapman \& Hall/CRC.
(in press).

 \item Nicolis, O and Nychka, D (2012). Reduced rank covariances for the analysis of environmental data.  In {\it Advanced Statistical Methods for the Analysis of Large Data-Sets}, Springer,  253--263.
  
\item  Nychka, D and Bo Li (2011). Discussion to:
A Statistical Analysis of Multiple Temperature Proxies: Are Reconstructions of Surface Temperature over the last 1000 Years Reliable?
  McShane and Wyner. {\it Annals of Applied Statistics }

\item 
D. Nychka and J.L. Anderson (2010). Data Assimilation. 
Chapter in {\it Handbook of Spatial Statistics} ed. and A Gelfand, P. Diggle, P. Guttorp and M. Fuentes . Chapman \& Hall/CRC. 
  

\item Nychka,D.  J. M. Restrepo and C. Tebaldi (2008). Uncertainty in Climate Predictions.  American Mathematical Society, Mathematics Awareness Month. 
 
 \item National Research Council. (2007). {\it Surface temperature reconstructions for the last 2,000 years}. National Academies Press.

\item Gilleland, E., D.  Nychka and U. Schneider (2006).  Spatial
models for the distribution of extremes.  In {\it Applications of
Computational Statistics in the Environmental Sciences: Hierarchical
Bayes and MCMC Methods} ed.  J.S. Clark and A. Gelfand, Oxford
University Press.

\item Tebaldi, C. and Nychka, D. (2004) Invited discussion to Calibrated
probabilistic mesoscale weather field forecasting: the geostatistical
output perturbation (GOP) method.  {\it Journal of the American
Statistical Association}. {\bf 99} 583--585. 

\item Nychka, D. and Tebaldi, C. (2002), Comment on `Calculation of Average,
Uncertainty Range and Reliability of Regional Climate Changes from AOGCM Simulations
via the Reliability Ensemble Averaging (REA) method' {\it Journal of Climate},
{\bf 16}, 883--884.

\item Nychka, D. (2000). Challenges in Understanding the Atmosphere. {\it Journal of the American Statistical Association}, {\bf 95},  972-975.  \\ See also {\it Statistics in the 21st Century}, ed., Raftery, A., Tanner, M. and Wells, M., Chapman and Hall/CRC, New York, 199-206.

\item {Hu, F., Hall, A. R.} and {Nychka, D.} (2000). A Nonparametric
Approach to
Stochastic Discount Factor Estimation. {\it Applying Kernel and
Nonparametric Estimation to Economic Topics}, eds. Fomby, T. and Hill,
R.C.,
JAI Press Inc., Stamford Connecticut, 155-178.

\item {Nychka, D.} (2000). Spatial Process Estimates as Smoothers.
{\it
Smoothing and Regression. Approaches, Computation and Application}, ed.
Schimek, M. G., Wiley, New York, 393-424.

\item {Nychka, D.} and {Saltzman, N.} (1998). Design of Air Quality
Monitoring Networks.  {\it Case Studies in Environmental Statistics}, ed. Nychka, D., Cox, L.,  Piegorsch, W., Lecture Notes in Statistics,  Springer-Verlag.


\item Holland, D., Saltzman, N., Cox, L. and Nychka, D. (1998).
Spatial Prediction of Sulfur Dioxide in the Eastern United States. {\it Proceedings of geoENV98}, Valencia, Spain, November 1998, Kluwer.

\item {Nychka, D., Yang, Q.} and {Royle, J. A.} (1997). Constructing
Spatial Designs Using Regression Subset Selection. {\it Statistics for the Environment-3: Pollution Assessment and Control}, eds. Barnett, V. and Turkman, K. F., Wiley, New York.

\item  {Nychka, D.} and {O'Connell, M.} (1996). Neural Networks in Applied Statistics - Discussion. {\it Technometrics}, {\bf 38}, 218-220.


\item {Bailey, B., Ellner S.} and {Nychka, D.} (1997).  Asymptotics and
 Applications of Local Lyapunov Exponents.  {\it Proceedings for
 the Fields/CRM Workshop: Nonlinear Dynamics and Time Series:
 Building a Bridge Between the Natural and Statistical Sciences},
 American Mathematical Society, 115-133.

\item { Nychka, D.} and { Cummins, D.} (1996). Comment on:
Eilers, P. and Marx, B.
Flexible Smoothing with B-splines and Penalties. {\it
Statistical Science}, {\bf 11}, 104-105.

\item {Nychka, D. W., Ellner, S.} and {Bailey, B. A.} (1995).
A Personal Overview of Nonlinear Time-series Analysis from a Chaos
Perspective - Discussion and Comments. {\it Scandinavian Journal of Statistics}, {\bf 22}, 433-435.

\item { Nychka, D.} (1994). Discussion to Epidemics: Models and
Data. {\it Journal of the Royal Statistical Society Series B}.

\item { Handcock, M.},{ Nychka, D.} and { Meier, K.} (1994).
Discussion to Kriging and Splines: An Empirical Comparison of Their
Predictive Performance. {\it Journal of the American Statistical
Association}, {\bf 89}, 401-403.


\item { Bloomfield, P., Brillinger, D. R., Nychka, D. W.} and {
Stolarski, R.}  (1988).  Appendix 1 Statistical Approaches to Ozone
Trend Detection. {\it Present State of Knowledge of the Upper
Atmosphere 1988: An Assessment Report NASA Reference Publication}
1208, ed. Watson, R. T., NASA.

\item { Bloomfield, P., Brillinger, D. R., Nychka, D. W.} and
{Stolarski, R.}  (1988).  Appendix 1 Statistical Issues.  {\it
Report of the NASA-WMO Ozone Trend Panel}, ed. Watson, R. T., NASA.

\end{enumerate}

\subsection*{Books}
\begin{itemize}
\item  {Nychka, D., Cox, L.} and  {Piegorsch, W.} (1998). {\it
Case Studies in Environmental Statistics},
Lecture Notes in Statistics,  Springer Verlag, New York.

\item {Berliner, L.M., Nychka, D.} and {Hoar, T.} (2000). {\it Statistics
for
Understanding the Atmosphere}, Springer Verlag, New York.
\end{itemize}

\subsection*{Educational Materials}
\begin{itemize}
\item Data Science 101  \\
Hands-on data analysis is used to introduce statistical concepts, teach how to program in R,  and solve practical data science problems. 
Taught twice as APPM2720 Spring 2015, Spring 2016, University of Colorado - Boulder.

 \item Spatial statistics short course \\ A four lecture series to introduce the concepts and extensions of modern spatial statistics and curve and surface fitting. 
 
\item { Nychka, D.} and { Boos, D.} (1993). {\it ${\cal S}$ Lab: A series
of labs
for teaching statistics and data analysis}. 
(A statistics lab manual and a set of
approximately 60 $ {\cal S}$ functions for teaching concepts of data
analysis
and probability.) 

\item { Nychka, D.} (1992). {\it Probability and Statistics for Engineers
and
Scientists.} North Carolina State University Video Extension Service.\\
 A series of 70, 50 minute video taped lectures covering the principles
of probability, basic statistics, regression and experimental design.
\end{itemize}

\subsection*{Software}
\begin{itemize}
 \item Nychka, D., Hammerling, D., Sain, S. Lenssen, N.  and Smirniotis, C. 
(2011-present). LatticeKrig: Multi-resolution Kriging based on Markov random fields \\
\verb+http://cran.r-project.org/web/packages/LatticeKrig+  \\
( $>$ 3.5K downloads since 7/2016)

 \item Nychka, D., Furrer, R., Paige, J., and Sain, S.
(2000-present). fields: Tools for Spatial Data //
\verb+http://cran.r-project.org/web/packages/fields+  \\
( $>$ 230K downloads since 7/2016)

\item Nychka, D., Bailey, B., Ellner, S., Haaland, P. and O'Connell, M.
(1996). FUNFITS Data Analysis and Statistical Tools for Estimating
Functions {\it North Carolina Institute of Statistics Mimeoseries }
 No.  2289 \\

%\item Nychka, D., Saltzman, N. and Royle, J. A., DI An Interface for Spatial
%Design (1996)  \\
\end{itemize}

%%%%%%%%%%%%%%%%%%%%%%%%%%%%%%%%%%%%%%%%%%%%%%%
\subsection*{Invited presentations}
Invited talks, seminars and short courses since October 2012 and grouped by fiscal year (FY): 10/1- 9/30 with
approximate audience size (\ ).

\begin{itemize}
\item Presentations FY17

{\it Pattern Scaling of Climate Models} \\
 November 2017, KAUST, Saudi Arabia (30) \\
 July 2017, Data Science and the Environment, Brest, FR (60) \\
 July 2017, University of Lancaster, UK (60) \\
 September 2017, Colorado School of Mines

{\it Estimating Curves and Surface}  \\
  April 2017, University of Maryland-Baltimore Campus, (60) 4 lectures 

{\it Spatial statistics} \\
  May 2017, University of Fudan, Shanghai, PRC, (35)  9 Lectures 

{\it Regional Climate and Extremes} \\
 October 2016, STATMOS Workshop on Extremes, College Station, PA (35) 

{\it Environmental Statistics at NCSU} \\
 October 2016, 75th Anniversary Department of Statistics,
  North Carolina State University, Raleigh, NC (70) 

{\it Solving Inverse Problems} \\
 October 2016, Reed College, (45) 

{\it Multi-resolution spatial methods: LatticeKrig} \\
 April 2017, CSU, (45) 


\item Presentations FY16

{\it Multi-resolution spatial methods: LatticeKrig} \\
 October 2015, KAUST, Saudi Arabia, (25) \\
 November 2015, Big Data and the Environment, Buenos Aires (50)  \\
 March 2016, Arizona State University, Tempe, AZ (40) \\
 March 2016, ETH�University Zurich, Zurich, CH (40) 
 
{\it Spatial Statistics}  \\
 July 2016, Environmental Analytics, NCAR (2 Lectures) (30) \\
 July 2016, Regional Climate Tutorial, NCAR (MMM)  (60) \\
 April 2016, Colorado School of Mines, Golden (25) \\
 June 2016,  R Bootcamp , NCAR (10) 

{\it Hierarchical Models}  \\
 July 2016, Beyond P-values, NCAR (25)  

{\it Regional Climate and Extremes} \\
 April 2016, Theme-of-the-Year, NCAR, Boulder (45) \\
 June 2016,  BIRS, Banff, CA (30) \\
 September 2016, Climate Informatics, NCAR (60) 

{\it Pattern Scaling of Climate Models} \\
 June 2016, 13th  International Statistics and Climate Conference, Canmore , CA (60) 

{\it Are Climate Models Built Using Statistics? }\\
 August 2016, Joint Meetings American Statistical Association, Chicago, (60) 

{\it Data analysis for extremes} \\
 August 2016, Tutorial CMIP5 Analysis Platform, NCAR (45) 2 Lectures  


\item Presentations FY15

{\it Uncertain Weather, Uncertain Climate} \\
  October 2014, University British Columbia, Vancouver, CA (50) \\
  
{\it Statistical inference for spatial data} \\
  November 2014, University of Kansas, Lawrence, KA (45) \\

{\it What would a statistician do with 10 seconds on a super computer?} \\
  November 2014, University of Kansas, Lawrence, KA (45) 

{\it Multi-resolution spatial methods: LatticeKrig} \\
  October, 2014, University of British Columbia, Vancouver, CA (45) \\
  November 2014, Michigan State University, E. Lansing, MI (45) \\
  April 2015, University of  Indiana (20) \\
  June 2015, Summer Research Conference on Statistics, Carolina Beach, NC (50) \\
  July 2015, Joint Statistical Meetings, Seattle (60) \\
  August 2015, University of Colorado-Denver, (25) \\
  September 2015, Colorado School of Mines, (25) 
 

{\it Regional climate informatics}  \\
  January 2015, Seismometrics, Valparaiso, Chile (50) 

{\it Spatial Statistics}  \\
  March 2015, Indian Statistical Institute, Kolkata,IN (2 Lectures) (30) \\
  July 2015, Data Analytics for Ecologists, NCAR (25)  \\
  July 2015, Regional Climate Short Course, NCAR (50) 

{\it A Statistical Excursion with DART} \\
  May 2015, STATMOS/Data Assimilation Short Course NCAR (20)


{\it Bayesian Hierarchical Models} \\
  April 2015, University of Indiana (20) 

{\it Regional Climate and Extremes} \\
  May 2015, Pacific Institute of Mathematics, Vancouver, (40) \\
  June 2015, BIRS, Banff, Alberta, (30) \\
  April 2016, Theme-of-the-Year, NCAR, Boulder (45) 

{\it Asymptotic theory for spatial methods} \\
  June 2015, Aalborg University,  Aalborg, Denmark (30) 

{\it HPC4Stats} \\
  September, 2015, STATMOS short course, University of Michigan, (25) 

{\it Pattern Scaling of climate models} \\
June 2016, 13th  International Statistics and Climate Conference, Canmore, CA 


\item  Presentations FY14

 {\it DART  and Ocean Data Assimilation}   \\
 October 2013, Role of the Oceans in Climate Uncertainty, BIRS, Banff,
 Alberta, CA (30)  

{\it Uncertain Weather, Uncertain Climate} \\
  October 2013, Department of Statistics, Brigham Young University,
  Provo, UT (60)  

{\it Regional Climate past, present and future}  \\
  November 2013, Royal Statistical Society and the American Statistical Association, London, UK (75)  

{\it Regional Climate, Extremes and Spatial Data} \\
   February, 2014, AAAS meeting, Chicago (10) 

{\it Multi-Resolution Spatial Methods for Large Data Sets}  \\
  November 2013, Exeter University, Exeter UK, (45) \\
  November 2013, CISL Work in Progress (30) \\
  February 2014, University of Chicago, Chicago, IL (35) \\
  February 2014, Harvard University, Boston, MA (30) \\
  April 2014,  SIAM/ASA Uncertainty Quantification, Savannah, GA, (4 Lectures)  (45) \\
  April 2014, National Science Foundation, Arlington, VA (30) \\
  May 2014, University of Glasgow, Scotland, UK (35) \\
  June 2014, Conference on Nonparametric Statistics for Big Data and
   Celebration to Honor Professor Grace Wahba, Madison, WI 
 
{\it Uncertain Weather, Uncertain Climate} \\
  May 2014, University of Glasgow, Scotland, (35) 
 

{\it Estimating Curves and Surfaces (4 Lectures)}  \\
  March 2014, KAUST, Saudi Arabia (45)  

{\it Statistical inference for spatial data} \\
  July 2014, SAMSI/IMAGe Summer Program: The International Surface Temperature Initiative 

{\it Reconstructing CO2 for the past 2000 years} \\
  August 2014, Joint Statistical Meetings, Boston, MA (20) (Invited poster session) 

{\it What would a statistician do with 10 seconds on a super computer?} \\
  August 2014, Joint Statistical Meetings, Boston, MA (50) 
  \item Presentations FY13

{\it Multi-Resolution Spatial Methods for Large Data Sets}  \\
  October 2012, U Arizona, Tucson, AZ (60) \\
  December 2012, Stanford U, Palo Alto, CA (40) \\
  February 2013, SAMSI Large Datasets, NCAR (50) \\
  March 2013, Iowa State U, Ames, IA, (60) \\
  May 2013, SIAM Data Mining Conference, Austin, TX (100) \\
  June 2013, International Meeting on Statistics and Climatology, Jeju, South Korea (75) \\
  July 2013 NSF Expeditions Workshop, Evanston, IL (40) \\
  September 2013,  Third Workshop on Bayesian Inference for Latent Gaussian Models,
                   Reykavik, Iceland (60) 
  
{\it Statistical Methods for Nonstationary Spatial Data}  \\
  December 2012, American Geophysical Union, San Francisco, CA (75) \\
  June 2013, International Meeting on Statistics and Climate, Jeju, South Korea (50) \\
  August 2013, American Statistical Association Annual Meeting, Montreal, CA (75) 

{\it Uncertain Weather, Uncertain Climate} \\
  March 2013, Invited University Lecture, U Toronto, CA (60) \\
  October 2013, Department of Statistics, Brigham Young University, Provo, UT (60)  




\end{itemize}








%%%%%%%
\subsection*{Advisement of Doctoral Students}
\begin{description}
\item[North Carolina State University]  \ \\
 Thomas Filloon, Statistics  May 1990 \\
 Kristen Meier,  Statistics   May 1990 \\
 Gary Kenney, Computer Science 1992 (committee co-chair)  \\
 Ju-Chin Huang, Economics and Statistics May 1994 (committee co-chair) \\
 Mark Graham, Electrical Engineering 1993 (committee co-chair) \\
 Bruce Elsheimer, Statistics December 1994 \\
 Barbara Bailey, Biomathematics May 1996 \\
 Jeffrey Andrew Royle, Statistics May 1996 \\
 David Cummins, Statistics May 1997 \\
 Jun Zhai, Statistics December 1997 \\
 Sarah Hardy, Statistics December 1998 \\
 Jung-min Baik, Statistics August 1999 (committee co-chair)

\item[NCAR] \ \\
 Eric Gilleland, Statistics, Colorado State Univ., December 2004 \\
Curtis Storlie, Statistics, Colorado State Univ., June 2005 (committee
member)\\ 
Daniel Cooley, Applied Mathematics, CU-Boulder, August 2005
 (co-chair) \\
Cari Kaufman, Statistics, Carnegie Mellon Univ., August 2006
(committee member) \\
Chris Paciorek, Statistics, Carnegie Mellon Univ.,  August 2006
Terry Lee, University of Victoria, March 2008 (external reviewer) \\
Suz Tolwinski-Ward, University of Arizona, July 2012 (committee member)\\
Whitney Wang, Purdue University, August 2017 (thesis advisor) \\
Collette Smirniotis, San Diego State University,  {\it current}  (committee member) 

\end{description}
\subsection*{Postdoctoral mentoring}
* indicates role as a co-mentor,   (latest organization of employement) \\
 Laura J. Stteinberg 1995 (Syracuse University)* \\
 Barbara Bailey,  1997  (San Diego State University)* \\
 Jeffrey Andrew Royle, 1997  (USGS Patuxent Wildlife Research Center)*\\
 Chris Wikle, 1997 (University of Missouri) *\\
 Montserrat Fuentes,1999 (Virginia Commonwealth University) \\
 Phillippe Naveau, 1998 - 2001 (Laboratoire des Sciences du Climat et l'Environnement) \\
 Claudia Tebaldi, 1998 - 2000 (National Center for Atmospheric Research)   \\
 Sarah Sreett, 2000-2003 (National Institute of Standards and Technology) \\
 Ulrike Schnieder 2003 - 2005 (Vienna University of Technology) \\
 Daniel Cooley, 2005 - 2007 ( Colorado State University)* \\
 Dorin Drignei  2005 -2007 (Oakland University) \\
 Reinhard Furrer 2002-2005 (Universit\"{a}it Z\"{u}rich)
 Bo Li 2007 - 2009 (University of Illinois) \\
 Cari Kaufman 2006 - 2008 (University of California- Berkeley) \\
 William Kleiber 2010-2012 (University of Colorado)* \\
 Matthew Heaton 2011-2013 (Brigham Young University)* \\
 Suz Tolwinski-Ward  2012 -2014 (Air Worldwide) \\
 Martin Tingley 2010 - 2011 (Netflix) \\
 Stacy Alexeeff 2013 - 2015 (Kaiser Permanente Northern California Division of Research) \\
 Kevin Dalmasse 2014-2017 (CNES  Observatoire Midi-Pyr�n�es, Toulose)*
 
%%%%%%%%%%%%
\subsection*{Grants and Contracts}
\begin{itemize}

\item  Scalable Statistical Validation and Uncertainty Quantification for Large Spatio?Temporal Datasets PI  2015 -  \$75,090.00

\item NSF Collaborative Research: Characterizing 21st Century Extremes for Engineering and Evaluating Robust Infrastructure Designs coPI 2016 - present

\item NSF Assessing and Improving the Scale Dependence of Ecosystem Processes in Earth System Models. coPI  2011-  \$896,674

\item 	NSF Mathematical Sciences: Multi-resolution lattice models and theory for spatial process estimators coPI 2007 - 2011 \$227,844

\item NSF Atmospheric Sciences	CMG Collaborative Research: Development of Bayesian Hierarchical Models to Reconstruct Climate Over the Past Millenium
PI 2007-2011 \$379,637

\item 	NSF Atmospheric Sciences The North American Regional Climate Change Assessment Program (NARCCAP)--Using Multiple GCMs and RCMs to Simulate Future Climates and Their Uncertainty coPI 2006- present \$941,779

\item NSF Mathematical Sciences:  A Statistics Program at the National
Center for Atmospheric Research. 2004-2007 PI  \$1,240,000

\item US EPA: Design Interface for Spatial Analysis of Monitoring
Networks. 2000, \$24,900

\item NSF Spatial Data and Scaling Methods for Assessment of Agricultural Impacts of Climate 
Managing Multiple Sources of Uncertainty Over Space. coPI 2000 \$725,000.00

\item NSF Mathematical Sciences:  A Statistics Program at the National
Center for Atmospheric Research. co-PI 1999-2004 \$3,000,000

\item NSF Mathematical Sciences: Process Design, Modeling and
Optimization in Electronics and Health Care Products. 1997, \$70,800

\item  Becton Dickinson Research Center: Nonparametric Regression and
Sequential Designs for Process Optimization. 1996, \$8,190

\item EPA Cooperative Agreement: Statistical Strategies for
Monitoring and Assessing
Environmental Changes and Effects. 1995, \$404,000 (co-PI)

\item    Becton Dickinson Research Center: New Directions in Process
Optimization. 1994, \$8,108

\item  NSF Mathematical Sciences: Estimation and Inference for
Nonlinear Systems. 1993, \$119,876

\item  NSF Atmospheric Sciences: Microscopic and Macroscopic
Approaches to Climate Dynamics.  PI 1992, \$54,137 

\item  NSF SCREMS program: Estimation and Inference for Noisy Chaotic
Systems. coI \$36,000 

\item  NSF Mathematical Sciences: Applications of Smoothing Splines
for Inference and Data Analysis. 1988, \$65,000

 \item  NCSU faculty development: Estimating Tumor Size Distributions
from Planar Cross Sections. 1984, \$3,494 

\item  NSF SCREMS program: Data Analysis with Generalized Splines.
1984, coI  \$45,000 


\end{itemize}


\end{document}
